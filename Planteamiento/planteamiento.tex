\chapter{Planteamiento del problema}\label{cap:01}
\thispagestyle{empty}


%-------------------------------------------
%	Párrafo(s) introductorio
%-------------------------------------------
\section{Introducción}
\textcolor{red}{Párrafo de resumen, que señala los puntos principales que se trataran en el planteamiento.}

%-------------------------------------------
%	Planteamiento del Problema
%-------------------------------------------
\section{Planteamiento del problema}

\textcolor{red}{Contiene información suficiente para persuadir al lector de que la interrogante de la investigación es relevante, oportuna, factible a desarrollar en un área específica del saber.}


%-------------------------------------------
%	 Importancia del estudio para el campo académico
%-------------------------------------------
\section{Importancia del estudio para el campo académico}

\textcolor{red}{Enuncia la contribución original al área específica del conocimiento o saber del estudio; describiendo por qué este estudio es original y beneficiará a la comunidad o profesión.}


%-------------------------------------------
%	 Naturaleza del estudio
%-------------------------------------------
\section{Naturaleza del estudio}

\textcolor{red}{Esta sección describe el diseño de investigación específico para responder a la pregunta de 
investigación(cómo se desarrollarán los objetivos o se abordará la pregunta de investigación) y se justifica su 
adecuación y selección.}


%-------------------------------------------
%	  Suposiciones
%-------------------------------------------
\section{Suposiciones}

\textcolor{red}{Esta sección debe enumerar lo que se supone que es cierto acerca de la información recopilada en la tesis: los supuestos que se aceptan para la tesis como metodológicos, teóricos o específicos del tema. Por ejemplo, los participantes respondieron las preguntas con honestidad o se asumen que las variables seleccionadas son las principales variables que afectarán los resultados. Cada suposición debe estar respaldada por una breve explicación.}

%-------------------------------------------
%	  Alcance
%-------------------------------------------
\section{Alcance y  limitaciones}

\textcolor{red}{Se declaran y establecen las principales limitaciones y delimitaciones del estudio. Las limitaciones no 
se encuentran bajo el control del investigador, mientras que las delimitaciones sí están bajo control. Por ejemplo, la 
falta de financiamiento es una limitación del estudio y las encuestas solo aplicadas en la zona rural de Puerto Plata es 
una delimitación de la investigación.}

%-------------------------------------------
%	Párrafo(s) de Cierre
%-------------------------------------------
\section{Cierre del capítulo}
\textcolor{red}{Párrafo o párrafos finales que sintetizan lo hallado y empalma con el siguiente capítulo.}
