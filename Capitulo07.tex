\chapter{Recomendaciones}
\thispagestyle{empty}
% \begin{flushright}
% \begin{minipage}{\mitadmas}
% \begin{quote}
% {\small\emph{``Pero existe otra razón para la gran reputación de la Matemática: la de que la Matemática ofrece a las ciencias naturales exactas un cierto grado de seguridad que sin ella no podrían alcanzar''}}
% \vspace{-.6cm}
% \begin{flushright}
% {\small\emph{\textbf{Albert Einstein (1879--1955)}}}
% \end{flushright}
% \end{quote}
% \end{minipage}
% \end{flushright}
% \vspace{1cm}

%-------------------------------------------
%	Recomendaciones
%-------------------------------------------
\section{Recomendaciones}

\textcolor{red}{\begin{enumerate}
	\item Las recomendaciones no son resúmenes o repeticiones o paráfrasis de lo que se sostiene en las premisas (capítulos-subcapítulos), ni tampoco son el resultado de una inferencia.
	\item Las recomendaciones son sugerencias que el investigador propone para su implementación y que están referidas a las conclusiones presentadas.
	\item Las recomendaciones se comunican como acciones: mientras que las conclusiones son aseveraciones, las recomendaciones son enunciados en forma de objetivos a realizar.
	\item Las recomendaciones versan sobre las implicaciones de la investigación, tanto a nivel de futuras investigaciones - necesidad de desarrollar más investigaciones- como también las implicaciones para la sociedad en general, recomendaciones de políticas públicas.
\end{enumerate}}
