\chapter{Metodología y plan de trabajo}\label{cap:04}
\thispagestyle{empty}

%-------------------------------------------
%	Introducción
%-------------------------------------------
\section{Introducción}

\textcolor{red}{El propósito de la investigación y la justificación del paradigma de investigación específico
están claramente establecidos. Preguntas de investigación e hipótesis}


%-------------------------------------------
%	Diseño de investigación
%-------------------------------------------
\section{Diseño de investigación}

\textcolor{red}{Presentación general del marco teórico del estudio y el propósito del estudio que luego
será desarrollado específicamente.}


%-------------------------------------------
%	Idoneidad/Adecuación del diseño
%-------------------------------------------
\section{Idoneidad/Adecuación del diseño} 
\textcolor{red}{El investigador debe asegurar al lector que el tipo de diseño de investigación está
justificado y apropiado para los resultados deseados.}


%-------------------------------------------
%	Técnicas y/o Instrumentos
%-------------------------------------------
\section{Técnicas y/o Instrumentos} 
\textcolor{red}{Descripción de las técnicas a utilizar y/o instrumentos creados para el caso específico
de la investigación.}


%-------------------------------------------
%	Validez y confiabilidad
%-------------------------------------------
\section{Validez y confiabilidad} 
\textcolor{red}{Tanto para un diseño cuantitativo como para un diseño cualitativo, se deben responder dos preguntas 
centrales referidas a la validez y confiabilidad de los datos, es decir, a la calidad de los mismos.
\begin{itemize}
     \item ¿Estoy realmente midiendo, registrando, capturando lo que pretendo? (Validez)
     \item ¿Es consistente y preciso todo aquello que estoy midiendo, registrando, capturando? (Confiabilidad)
\end{itemize}}

%-------------------------------------------
%	Plan de Trabajo
%-------------------------------------------
\section{Plan de Trabajo} 
\textcolor{red}{Se debe presentar un cronograma con fechas y períodos de tiempo a utilizar para el desarrollo del 
trabajo.}

%-------------------------------------------
%	Párrafo(s) de Cierre
%-------------------------------------------
\section{Cierre del capítulo} 
\textcolor{red}{Párrafo o párrafos finales que sintetizan lo hallado.}
