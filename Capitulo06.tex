\chapter{Conclusiones}
\thispagestyle{empty}
% \begin{flushright}
% \begin{minipage}{\mitadmas}
% \begin{quote}
% {\small\emph{``Pero existe otra razón para la gran reputación de la Matemática: la de que la Matemática ofrece a las ciencias naturales exactas un cierto grado de seguridad que sin ella no podrían alcanzar''}}
% \vspace{-.6cm}
% \begin{flushright}
% {\small\emph{\textbf{Albert Einstein (1879--1955)}}}
% \end{flushright}
% \end{quote}
% \end{minipage}
% \end{flushright}
% \vspace{1cm}

%-------------------------------------------
%	Conclusiones
%-------------------------------------------
\section{Conclusiones}

\textcolor{red}{\begin{enumerate}
	\item Las conclusiones son oraciones aseverativas, enunciados, proposiciones, afirmaciones nuevas que se derivan de lo que se ha expuesto en las premisas (capítulos y subcapítulos).
	\item Son una lista de afirmaciones que se derivan lógicamente (proceso de inferencia) de todo lo que se ha afirmado en los capítulos. En este sentido, constituyen el balance final de todo el trabajo.
	\item Las conclusiones no son resúmenes, síntesis o repeticiones o paráfrasis de lo que se sostiene en las premisas (capítulos-subcapítulos), ni tampoco comentarios.
	\item No son recomendaciones y, por lo tanto, no son sugerencias a implementar.
\end{enumerate}}
