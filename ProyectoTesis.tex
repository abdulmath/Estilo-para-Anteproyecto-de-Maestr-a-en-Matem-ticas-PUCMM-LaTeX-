%%%%% Estilo creado para formatear el  Anteproyecto de Maestría en Matemática de la PUCMM
% Creado por Dr. Abdul Abner Lugo Jimenez
% Fecha 29 de Noviembre de 2022
\documentclass[letterpaper,12pt,oneside]{book}
\usepackage[spanish,activeacute]{babel}
\usepackage[utf8]{inputenc}
\usepackage[T1]{fontenc}
\usepackage{calligra,mathrsfs,mathptmx}
\usepackage{url,times,color}
\usepackage{amssymb,amsmath,amsfonts,dsfont,latexsym,amsthm,esvect}
\usepackage{graphicx,graphics,subfigure,wrapfig}
\usepackage{multirow,multicol,rotating,float}
\usepackage{colortbl}
\usepackage[Conny]{fncychap}
\usepackage{fancyhdr}
\renewcommand{\baselinestretch}{2}
\parindent=10mm
\newlength{\mitad}
\setlength{\mitad}{0.5\textwidth}
\newlength{\mitadmas}
\setlength{\mitadmas}{0.68\textwidth}
\usepackage[letterpaper,top=2.5cm,right=2cm,left=3cm,paperwidth=21.6cm,paperheight=27.9cm,
textwidth=16.6cm,textheight=22.9cm]{geometry}
\usepackage{setspace,boiboites}
\usepackage[nottoc,notlot,notlof]{tocbibind}
\usepackage{varwidth,stackrel}
\usepackage{xparse,tcolorbox,tikz}
\usepackage{makeidx}
\usepackage{algorithm,algorithmic}
\usepackage{apacite}
%%%%%%%%%%%%%%%%%%%%%%%%%%%%%%%%%%%%%%%%%%%%%%%%%%%%%%%%%%%%%%%%%%%
% Background de la Portada
\usepackage[pages=some,placement=top,opacity=1]{background}
\SetBgContents{\tikz{\draw[azulPUCMM,fill] (0,0) rectangle (.5,3);}}
\SetBgHshift{-23.5}
\SetBgVshift{0.5}
%%%%%%%%%%%%%%%%%%%%%%%%%%%%%%%%%%%%%%%%%%%%%%%%%%%%%%%%%%%%%%%%%%%%
\makeindex
%%%%%%%%%%%%%%%%%%%%%%%%%%%%%%%%%%%%%%%%%%%%%%%%%%%%%%%%%%%%%%%%%%%%
% \definecolor{grey}{rgb}{0.43,0.47,0.61} 
\definecolor{azulPUCMM}{RGB}{0,93,170}
% \newrgbcolor{bluepast}{0.6 1 1}
% \newrgbcolor{yellowpast}{1 0.8 0.2}
% \newrgbcolor{lavendar}{0.8 0.6 1}
% \newrgbcolor{greenpast}{0.565 0.6375 0.25}
% \newrgbcolor{indigo}{0.3098 0.5059 0.7412}
% \newrgbcolor{wine}{0.6471 0 0.1294}
% \newrgbcolor{orange}{0.9019 0.5372 0}
% \newrgbcolor{skin}{0.9804 0.74901 0.5608}
% \newrgbcolor{blanco}{1 1 1}
%%%%%%%%%%%%%%%%%%%%%%%%%%%%%%%%%%%%%%%%%%%%%%%%%%%%%%%%%%%%%%%%%%%%
\decimalpoint
\addto\captionsspanish{%
\def\figurename{{\small Figura}}%
\def\tablename{Tabla}%
\def\contentsname{Índice General}%
\def\listtablename{Índice de Tablas}%
\def\bibname{Referencias Bibliográficas}
}
%%%%%%%%%%%%%%%%%%%%%%%%%%%%%%%%%%%%%%%
%\renewcommand{\bibname}{Referencias Bibliográficas}
%%%%%%%%%%%%%%%%%%%%%%%%%%%%%%%%%%%%%%%%%%%%%%%%%%%%%%%%%%%%%%%%%%%%
\pagestyle{fancy}
\fancyhf{}
\fancyhead[LO]{\nouppercase{\leftmark}}
\fancyhead[R]{\nouppercase{\rightmark}}
\fancyhead[RO]{\thepage}
\renewcommand{\chaptermark}[1]{\markboth{\textbf{\thechapter. #1}}{}}
\renewcommand{\sectionmark}[1]{\markright{\textbf{\thesection. #1}}}
\renewcommand{\headrulewidth}{0.6pt}
%\renewcommand{\footrulewidth}{0.6pt}
\setlength{\headheight}{1\headheight}
%%%%%%%%%%%%%%%%%%%%%%%%%%%%%%%%%%%%%%%%%%%%%%%%%%%%%%%%%%%%%%%%%%%%%%%%%%%%%
\def\membrete{
\begin{flushleft}
\hspace{-6cm}\includegraphics[]{./figuras/MarcaPUCMM.png}
\end{flushleft}
\begin{center}
\textsc{Vicerrectorado de Postgrado\\
Escuela de Ciencias Básicas y Exactas\\
Maestría en Matemática}
\end{center}}
\onehalfspacing
%%%%%%%%%%%%%%%%%%%%%%%%%%%%%%%%%%%%%%%%%%%%%%%%%%%%%%%%%%%%%%%%%%%%%%%%%%%%%
% Este comando evita que aparezcan hojas de sobra.
\renewcommand{\cleardoublepage}{\clearpage\if@twoside\ifodd\c@page\else\hbox{}
\thispagestyle{empty}
\newpage\if@twocolumn\hbox{}\newpage\fi\fi\fi}
%%%%%%%%%%%%%%%%%%%%%%%%%
%Teoremas, Corolarios, Definiciones, etc%
%%%%%%%%%%%%%%%%%%%%%%%%%
\renewcommand{\qedsymbol}{$\blacksquare$}
%%%%%%%%%%%%%%%%%%%%%%%%%%%%%%%%%%%%%%%%%%%%%%%%%%%%%%%%%%%%%%%%%%%%%%%

%%%%%%%%%%%%%%%%%%%%%%%%%%%%%%%%%%%%%%%%%%%%%%%%%%%%%%%%%%%%%%%%%%%%%%%%
%\theorembodyfont{\normalfont}
%\theoremheaderfont{\scshape\large}
% COLORES personales---------------------------------------------------
    \definecolor{colortitulo}{RGB}{11,17,79} % 
    \definecolor{colordominante}{RGB}{11,17,79}
    \definecolor{colordominanteF}{RGB}{219,68,14}
    \definecolor{colordominanteD}{RGB}{137,46,55}
    \definecolor{mostaza}{RGB}{231,196,25}
    \definecolor{amarilloM}{RGB}{248,199,90}
    \definecolor{amarilloD}{RGB}{251,237,121}
    \definecolor{azulF}{rgb}{.0,.0,.3}
    \definecolor{grisD}{rgb}{.3,.3,.3}
    \definecolor{grisF}{rgb}{.6,.6,.6}
    \definecolor{grisamarillo}{RGB}{248,248,245} 
    \definecolor{miverde}{RGB}{44,162,67}
    \definecolor{verdep}{RGB}{166,206,58}
    \definecolor{verdencabezado}{RGB}{166,206,58}
    \definecolor{verdeF}{RGB}{5,92,8}
    \colorlet{mygray}{black!20}
    \newcommand{\verde}{\color{miverde}}
% Fin COLORES personales-------------------------------------------------
\decimalpoint
%\newtheorem{alg}{Algoritmo}
\newboxedtheorem[boxcolor=blue, background=grisF!15, titlebackground=blue!20,
                   titleboxcolor = black]{teo}{Teorema}{thCounter} 
\newboxedtheorem[boxcolor=verdep, background=blue!5, titlebackground=blue!20,
                   titleboxcolor = black]{lema}{Lema}{leCounter}
\newboxedtheorem[boxcolor=purple, background=blue!2, titlebackground=blue!25,
                   titleboxcolor = black]{coro}{Corolario}{coCounter}
\newboxedtheorem[boxcolor=orange, background=blue!5, titlebackground=blue!20,
                   titleboxcolor = black]{defi}{Definición}{dfCounter}
\newboxedtheorem[boxcolor=verdeF, background=blue!3, titlebackground=blue!25,
                   titleboxcolor = black]{ejem}{Ejemplo}{ejCounter}
\newboxedtheorem[boxcolor=colordominanteD, background=blue!5, titlebackground=blue!20,
                   titleboxcolor = black]{ejer}{Ejercicio}{ejeCounter}
\newboxedtheorem[boxcolor=yellow, background=blue!10, titlebackground=blue!15,
                   titleboxcolor = black]{obs}{Observación}{obsCounter}                 
\newboxedtheorem[boxcolor=miverde, background=blue!20, titlebackground=blue!15,
                   titleboxcolor = black]{propo}{Proposición}{propoCounter}
\newboxedtheorem[boxcolor=azulF, background=red!10, titlebackground=red!5,
                   titleboxcolor = black]{nota}{Nota}{}
%%%%%%%%%%%%%%%%%%%%%%%%%%%%%%%%%%%%%%%%%%%%%%%%%%%%
\setcounter{secnumdepth}{5} % para que ponga 1.1.1.1 en subsubsecciones
\setcounter{tocdepth}{5}
%%%%%%%%%%%%%%%%%%%%%%
%Cuerpo del Documento%
%%%%%%%%%%%%%%%%%%%%%%
\begin{document}
\frontmatter
\selectlanguage{spanish}
\BgThispage
\begin{titlepage}
\begin{flushright}
\begin{minipage}{12cm}
\begin{center}
\membrete
\end{center}
\end{minipage}
\end{flushright}
\vfill
\begin{flushright}
\begin{minipage}{12cm}
\begin{center}
\vskip 1cm \rule{12cm}{3pt}\\[1mm]
{\Large Proyecto de Maestría}\\[1mm]
{\Large \textsc{Aquí va el título del proyecto}\\[1mm]
\textsc{Otra línea por si es necesario se va ajustanto}}
\rule{12cm}{3pt}\\[6mm]
\end{center}
\end{minipage}
\end{flushright}
\vfill
\begin{flushright}
\begin{minipage}{12cm}
\begin{center}
\textsc{Proyecto de Maestría presentado a la Pontificia Universidad Católica Madre y Maestra, en 
cumplimiento parcial de los requisitos para iniciar el trabajo final de grado a Magister en Matemática.}\\[2cm]
\textsc{{\bf Autor}: Lcdo. Manuel Rosario}\\[2mm]
\textsc{{\bf Asesor}: Dr. Abdul Abner Lugo Jiménez.}
\end{center}
\end{minipage}
\end{flushright}
\vfill
\begin{flushright}
\begin{minipage}{12cm}
\begin{center}
Santiago de los Caballeros, \today.
\end{center}
\end{minipage}
\end{flushright}
\end{titlepage}




\thispagestyle{empty}
\vspace*{\fill}
\begin{flushright}
\begin{minipage}{\mitad}
{\Large\bf \em Dedicatoria}

\smallskip
En caso de ser necesario.
\end{minipage}
\end{flushright}
\vspace*{\fill}
\newpage
\thispagestyle{empty}
\vspace*{\fill}
\begin{center}{\Large\bf \em Agradecimientos}\end{center}

\vfill
\begin{itemize}
     \item Aquí lo que desees colocar, es algo personal si así lo desea.
     \end{itemize}
\vspace*{\fill}
\begin{flushright}
{\fontfamily{calligra}\fontsize{20}{0}\selectfont{Autor}}
\end{flushright}
\vspace*{\fill}
\newpage
\vfill
\thispagestyle{empty}
\begin{center}
{\large\bf Título del trabajo}\\
{\textit{(Título en ingles)}}

\vspace{1cm}
por\\
{\small Autor}
\end{center}

\begin{center}
\textbf{Resumen}
\end{center}

Acá se agrega un resumen del trabajo 

\noindent{\bf Palabras Claves:} algunas palabras claves según la investigación.

\begin{center}
\textbf{Abstract}
\end{center}
El mismo resumen pero en ingles

\noindent{\bf Keywords:} las mismas palabras claves en ingles.
\vfill
\newpage 

\tableofcontents
\listoffigures
\listoftables
\mainmatter
\chapter{Planteamiento del problema}\label{cap:01}
\thispagestyle{empty}


%-------------------------------------------
%	Párrafo(s) introductorio
%-------------------------------------------
\section{Introducción}
\textcolor{red}{Párrafo de resumen, que señala los puntos principales que se trataran en el planteamiento.}

%-------------------------------------------
%	Planteamiento del Problema
%-------------------------------------------
\section{Planteamiento del problema}

\textcolor{red}{Contiene información suficiente para persuadir al lector de que la interrogante de la investigación es relevante, oportuna, factible a desarrollar en un área específica del saber.}


%-------------------------------------------
%	 Importancia del estudio para el campo académico
%-------------------------------------------
\section{Importancia del estudio para el campo académico}

\textcolor{red}{Enuncia la contribución original al área específica del conocimiento o saber del estudio; describiendo por qué este estudio es original y beneficiará a la comunidad o profesión.}


%-------------------------------------------
%	 Naturaleza del estudio
%-------------------------------------------
\section{Naturaleza del estudio}

\textcolor{red}{Esta sección describe el diseño de investigación específico para responder a la pregunta de 
investigación(cómo se desarrollarán los objetivos o se abordará la pregunta de investigación) y se justifica su 
adecuación y selección.}


%-------------------------------------------
%	  Suposiciones
%-------------------------------------------
\section{Suposiciones}

\textcolor{red}{Esta sección debe enumerar lo que se supone que es cierto acerca de la información recopilada en la tesis: los supuestos que se aceptan para la tesis como metodológicos, teóricos o específicos del tema. Por ejemplo, los participantes respondieron las preguntas con honestidad o se asumen que las variables seleccionadas son las principales variables que afectarán los resultados. Cada suposición debe estar respaldada por una breve explicación.}

%-------------------------------------------
%	  Alcance
%-------------------------------------------
\section{Alcance y  limitaciones}

\textcolor{red}{Se declaran y establecen las principales limitaciones y delimitaciones del estudio. Las limitaciones no 
se encuentran bajo el control del investigador, mientras que las delimitaciones sí están bajo control. Por ejemplo, la 
falta de financiamiento es una limitación del estudio y las encuestas solo aplicadas en la zona rural de Puerto Plata es 
una delimitación de la investigación.}

%-------------------------------------------
%	Párrafo(s) de Cierre
%-------------------------------------------
\section{Cierre del capítulo}
\textcolor{red}{Párrafo o párrafos finales que sintetizan lo hallado y empalma con el siguiente capítulo.}

\chapter{Planteamiento del problema}\label{cap:02}
\thispagestyle{empty}

%-------------------------------------------
%	Párrafo(s) introductorio
%-------------------------------------------
\section{Introducción}
\textcolor{red}{Párrafo de resumen, a modo de transición, del capítulo anterior y que señala los puntos principales que 
se trataran en la revisión de la literatura.}

%-------------------------------------------
%	Antecedentes
%-------------------------------------------
\section{Antecedentes}

\textcolor{red}{Se describe el marco del problema en su contexto (cómo ha evolucionado el problema y qué se ha 
investigado) y la importancia y necesidad de la investigación, contexto y perspectiva.}


%-------------------------------------------
%	Documentación
%-------------------------------------------
\section{Tipo de revisión y justificación de la revisión}
\textcolor{red}{Descripción de las estrategias de búsqueda, fuentes de la documentación, tipos de materiales, etc. incluidos en la revisión de la literatura.}
 

%-------------------------------------------
%	Párrafo(s) de Cierre
%-------------------------------------------
\section{Cierre del capítulo}
\textcolor{red}{Párrafo o párrafos finales que sintetizan lo hallado y empalma con la siguiente sección.  ¿Cuáles son los principales puntos (definiciones, constructos, metodológicos, hallazgos) que se desprenden de la revisión crítica de la bibliografía esencial para la investigación?}

\chapter{Objetivos del proyecto}\label{Cap:03}
\thispagestyle{empty}

%-------------------------------------------
%	Introducción
%-------------------------------------------
\section{Introducción}

\textcolor{red}{Descripción del propósito de la investigación y qué identificar, de manera sucinta, el método de 
investigación, las variables, actores, o caso del estudio.}

%-------------------------------------------
%	 Pregunta e hipótesis de investigación
%-------------------------------------------
\section{Preguntas e hipótesis de investigación}

\textcolor{red}{\begin{enumerate}
	\item La pregunta, o preguntas de investigación, es una interrogante que todavía no han sido resuelta o se 
encuentran en discusión; son preguntas claras y precisas; preguntas que pueden resolverse en función de una metodología; 
y que responden a un tema de investigación muy bien delimitado y viable.
\end{enumerate}}

%-------------------------------------------
%	Objetivo General
%-------------------------------------------
\section{Objetivo General}

\textcolor{red}{Presentación general del marco teórico del estudio y el propósito del estudio que luego
será desarrollado específicamente.}

%-------------------------------------------
%	Objetivos Específicos
%-------------------------------------------
\section{Objetivos Específicos}

\textcolor{red}{\begin{enumerate}
	\item Diseños No Experimentales: Longitudinales, Transeccionales o Transversales, Cohortes, Encuestas.
	\item Diseños Experimentales: Pre-Experimentales, Experimentales, Puros, Cuasiexperimentales, Correlacionales y Ex post facto, Población--participantes.
	\item Tipos de Investigación Cuantitativa: Exploratoria, Descriptiva, Correlacional, Explicativa, Proyectiva.
\end{enumerate}} 

%-------------------------------------------
%	Párrafo(s) de Cierre
%-------------------------------------------
\section{Cierre del capítulo} 
\textcolor{red}{Párrafo o párrafos finales que sintetizan lo hallado y empalma con el siguiente capítulo.}
 









\input{./Plan/plan}
%%%%%%%%Comentar \nocite para agregar sus nuevas citas, es solo un ejemplo.
\nocite{CFN1950,BN1966,R1969,L2021,BCO1981,W1981,HSSKSC1997,C1997,J2002,SJ2004,G2005,AO2010,CL2015,MLR2016}
\backmatter
%-----------------------------------------------------------------------
% Bibliografía
%-----------------------------------------------------------------------
\thispagestyle{empty}
% \addcontentsline{toc}{section}{\color{azulF} Referencias Bibliograficas}
\bibliographystyle{apacite}
\bibliography{./Bibliografia/bibliografia}
%------------------------------------------------------------------------------------- 
\end{document}
