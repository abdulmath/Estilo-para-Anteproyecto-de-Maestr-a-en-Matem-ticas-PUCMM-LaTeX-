\chapter{Planteamiento del problema}\label{cap:02}
\thispagestyle{empty}

%-------------------------------------------
%	Párrafo(s) introductorio
%-------------------------------------------
\section{Introducción}
\textcolor{red}{Párrafo de resumen, a modo de transición, del capítulo anterior y que señala los puntos principales que 
se trataran en la revisión de la literatura.}

%-------------------------------------------
%	Antecedentes
%-------------------------------------------
\section{Antecedentes}

\textcolor{red}{Se describe el marco del problema en su contexto (cómo ha evolucionado el problema y qué se ha 
investigado) y la importancia y necesidad de la investigación, contexto y perspectiva.}


%-------------------------------------------
%	Documentación
%-------------------------------------------
\section{Tipo de revisión y justificación de la revisión}
\textcolor{red}{Descripción de las estrategias de búsqueda, fuentes de la documentación, tipos de materiales, etc. incluidos en la revisión de la literatura.}
 

%-------------------------------------------
%	Párrafo(s) de Cierre
%-------------------------------------------
\section{Cierre del capítulo}
\textcolor{red}{Párrafo o párrafos finales que sintetizan lo hallado y empalma con la siguiente sección.  ¿Cuáles son los principales puntos (definiciones, constructos, metodológicos, hallazgos) que se desprenden de la revisión crítica de la bibliografía esencial para la investigación?}
