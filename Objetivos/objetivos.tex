\chapter{Objetivos del proyecto}\label{Cap:03}
\thispagestyle{empty}

%-------------------------------------------
%	Introducción
%-------------------------------------------
\section{Introducción}

\textcolor{red}{Descripción del propósito de la investigación y qué identificar, de manera sucinta, el método de 
investigación, las variables, actores, o caso del estudio.}

%-------------------------------------------
%	 Pregunta e hipótesis de investigación
%-------------------------------------------
\section{Preguntas e hipótesis de investigación}

\textcolor{red}{\begin{enumerate}
	\item La pregunta, o preguntas de investigación, es una interrogante que todavía no han sido resuelta o se 
encuentran en discusión; son preguntas claras y precisas; preguntas que pueden resolverse en función de una metodología; 
y que responden a un tema de investigación muy bien delimitado y viable.
\end{enumerate}}

%-------------------------------------------
%	Objetivo General
%-------------------------------------------
\section{Objetivo General}

\textcolor{red}{Presentación general del marco teórico del estudio y el propósito del estudio que luego
será desarrollado específicamente.}

%-------------------------------------------
%	Objetivos Específicos
%-------------------------------------------
\section{Objetivos Específicos}

\textcolor{red}{\begin{enumerate}
	\item Diseños No Experimentales: Longitudinales, Transeccionales o Transversales, Cohortes, Encuestas.
	\item Diseños Experimentales: Pre-Experimentales, Experimentales, Puros, Cuasiexperimentales, Correlacionales y Ex post facto, Población--participantes.
	\item Tipos de Investigación Cuantitativa: Exploratoria, Descriptiva, Correlacional, Explicativa, Proyectiva.
\end{enumerate}} 

%-------------------------------------------
%	Párrafo(s) de Cierre
%-------------------------------------------
\section{Cierre del capítulo} 
\textcolor{red}{Párrafo o párrafos finales que sintetizan lo hallado y empalma con el siguiente capítulo.}
 








